%% Packages
% Formatting 
\usepackage[margin=2cm,headheight=35pt]{geometry}
\usepackage{textcomp}
\usepackage{fancyvrb}
\usepackage{fancyhdr}
\usepackage{listings}
\usepackage{enumitem}
\usepackage{collectbox}
\usepackage[usenames,dvipsnames]{xcolor}
\usepackage{caption}
\usepackage{subcaption}
\usepackage[skip=0pt, indent=0pt]{parskip}
\usepackage{graphicx}
\usepackage{wrapfig}
\usepackage{siunitx}
\usepackage{tikz}
\usepackage{tcolorbox}
\usepackage{algorithmic}

% Math
\usepackage{cancel}
\usepackage{amsmath, amsfonts, mathtools, amsthm, amssymb}
\usepackage{mathtools}
\usepackage{witharrows}

% Thrm
\usepackage{thmtools}
\usepackage[framemethod=TikZ]{mdframed}
\mdfsetup{skipabove=1em,skipbelow=0em}

% Definition box
\theoremstyle{definition}
\declaretheoremstyle[
    headfont=\bfseries\sffamily\color{ForestGreen!70!black}, bodyfont=\normalfont,
    notebraces={}{},
    headpunct={},
    mdframed={
            linewidth=2pt,
            rightline=false, topline=false, bottomline=false,
            linecolor=ForestGreen, backgroundcolor=ForestGreen!8,
            nobreak=false
        }
]{thmgreen2box}
\declaretheorem[style=thmgreen2box, name=Definition, numbered=no]{definition}

% Example box
\declaretheoremstyle[
    headfont=\bfseries\sffamily\color{NavyBlue!70!black}, bodyfont=\normalfont,
    notebraces={}{},
    headpunct={},
    mdframed={
            linewidth=2pt,
            rightline=false, topline=false, bottomline=false,
            linecolor=NavyBlue, backgroundcolor=NavyBlue!5,
            nobreak=false
        }
]{thmbluebox}
\declaretheorem[style=thmbluebox, numbered=no, name=Example]{eg}

% Theorem box
\declaretheoremstyle[
    headfont=\bfseries\sffamily\color{RawSienna!70!black}, bodyfont=\normalfont,
    notebraces={}{},
    headpunct={},
    mdframed={
            linewidth=2pt,
            rightline=false, topline=false, bottomline=false,
            linecolor=RawSienna, backgroundcolor=RawSienna!8,
            nobreak=false
        }
]{thmred2box}
\declaretheorem[style=thmred2box, name=Theorem, numbered=no]{theorem}

% Remark Box 
\declaretheoremstyle[
    headfont=\bfseries\sffamily\color{TealBlue!70!black}, bodyfont=\normalfont,
    mdframed={
            linewidth=2pt,
            rightline=false, topline=false, bottomline=false,
            linecolor=TealBlue,
            nobreak=false
        }
]{thmblueline}
\declaretheorem[style=thmblueline, numbered=no, name=Remark]{remark}

% Citation and linking
\usepackage{hyperref}
\usepackage{cite}
\usepackage{cleveref}

% Misc
\usepackage{lipsum}

%% Exponentials  
\newcommand{\e}[0]{\mathrm{e}}                      % math e 
\newcommand{\expnum}[2]{{#1}\mathrm{e}{#2}}         % exponential number

%% Vector notation
\newcommand{\dbar}[1]{\overline{\overline{#1}}}     % double overbar
\newcommand{\var}{\mathrm{Var}}                     % variance
\newcommand{\Dbar}[1]{\dot{\bar{#1}}}               % double bar
\newcommand{\Ddbar}[1]{\ddot{\bar{#1}}}             % dot over bar
\newcommand{\Dtilde}[1]{\dot{\tilde{#1}}}           % dot over tilde

%% Formatting
% \setlength{\parindent}{0pt}
\renewcommand{\thesubsection}{\alph{subsection}.}
\pagestyle{fancy}
\fancyhf{}
\fancyfoot[C,CO]{\thepage}

% I'm not sure what this is
% \definecolor{codegreen}{rgb}{0,0.6,0}
% \definecolor{codegray}{rgb}{0.5,0.5,0.5}
% \definecolor{codepurple}{rgb}{0.58,0,0.82}
% \definecolor{backcolour}{rgb}{0.95,0.95,0.92}
% \lstdefinestyle{mystyle}{
%     backgroundcolor=\color{backcolour},   
%     commentstyle=\color{codegreen},
%     keywordstyle=\color{magenta},
%     numberstyle=\tiny\color{codegray},
%     stringstyle=\color{codepurple},
%     basicstyle=\ttfamily\footnotesize,
%     breakatwhitespace=false,         
%     breaklines=true,                 
%     captionpos=b,                    
%     keepspaces=true,                 
%     numbers=left,                    
%     numbersep=5pt,                  
%     showspaces=false,                
%     showstringspaces=false,
%     showtabs=false,                  
%     tabsize=2
% }
% \lstset{style=mystyle}
