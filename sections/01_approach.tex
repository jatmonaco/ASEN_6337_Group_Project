\section{Approach and Methodology}
\textit{This is where you should describe the data analysis approach. Formulate your remote sensing data analysis problem using methods and concepts you learned in the class. }

This is a multi-class classification problem with many labeled training examples where the form of the class distributions are not known. This makes supervised discriminative classifiers the most appropriate family of tools. While tree-based methods and support vector machines are viable options, convolutional neural networks (CNNs) are the obvious tool for image-based classification.

\subsection*{Learning from Previous Submissions}
Our group did not have expertise in designing and training CNNs before this project. Fortunately, the competition's \textit{Kaggle} community was extremely collaborative and communicative in nature, with many top submissions publishing and discussing their solutions in detail for others to reproduce their results. We began by exploring existing solutions to get an idea of the tools that others found useful and to better understand a variety of approaches used. 

A big takeaway from exploring \textit{Kaggle} was that the Python-based \texttt{pytorch} was clearly the most popular machine learning framework used among submissions, and we decided to develop a \texttt{pytorch}-based solution as well. We also incorporated helper functions that have seemed to have been widely disseminated among solutions; any functions copied from \textit{Kaggle} include a link to the submission from which it was taken. 

\subsection*{Neural Network Architecture}
The input of the CNN has to take in the two-dimensional images. A single pixel can be a member to multiple classes, so a single mask that outputs one assignment per pixel is not sufficient; the CNN must output four two-dimensional masks the same size as the input. Our initial approach involved four different networks, each specializing in one class, but that risked missing between-class correlations. 

Another choice in architecture are the CNN's optimizer and loss function during training. 

Additionally, the number of fully-connected layers and neurons per layer, convolution layers and their kernel sizes, pooling layers and their pooling sizes and methods, and the ways layers are connected are all design choices of the network. 

\subsection*{Image Pre-Processing}
Another aspect to consider is what the inputs are to neural networks. The images are relatively large, at \(1400 \times 2100\) pixels per RGB channel. When using the full-channel, full-resolution images as inputs, we found a single training epoch took over 3 hours to complete for even simple CNNs. Most competition entries took a few dozen epochs to complete training, which makes rapid model development infeasible with the full-resolution images. Fortunately, the competition is evaluated on masks of images that are down sampled by a factor of four in each dimension. To decrease training time, we decided to down sample the training images and masks similarly.

Because the images were simply clouds over water, three image channels seemed egregious when a single channel could more concisely capture the cloud/water contrast. We decided to further reduce the three-channel image to a single channel via PCA. We tried for reduce via K-means as well, but found it often classified glare and clouds similarly, even when a high number of groups were used. After the images underwent PCA then down sampling, simple CNNs had a training epoch time of just \(\sim25\) minutes. 

% \subsection{Our First Approach: Divide and Concur}
% As a first step towards a solution, and because of the overlapping nature of the labels, we decided to start off by creating a different classifier for each class; i.e., a sugar classifier, a gravel classifier, a flower classifier, and a fish classifier. Each member of the team was assigned a class and created a classifier on their own using \texttt{pytorch} as the neural network tool set. This allowed us to get to know these new tools and data, come up with ideas on our own, and reconvene afterwards to see if there was any convergent thinking or useful tips before synthesizing a more general model. 
% %
% \subsection*{James' Approach for Sugar}

% \subsection{Other Ideas Considered}
% Additional ideas include: 
% \begin{itemize}
%     \item Creating additional synthetic training images by applying different combinations of affine transformations on the supplied training images (rotating, mirroring, stretching, wrapping). 
%     \item Creating a dedicated model to create labels for each class (4 models total).
% \end{itemize}