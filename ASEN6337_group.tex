%& -job-name=HW_template
\documentclass[12pt]{scrartcl}
\input{header.tex}
\hypersetup{pdftitle={ASEN 6337 Group Project}}

%% --- Title and Authors --- %% 
\title{\Huge Understanding Clouds from Satellite Images 
        \\ \huge Group: Why So Cirrus? \\\large ASEN 6337 Group Project Report  \\ Fall '24}
\author[1]{J. Monaco}
\author[1]{Maggie Zheng}
\author[1]{Kawther Rouabhi}
\author[1]{AJ Cuddeback}
\affil[1]{University of Colorado Boulder, Smead Aerospace Engineering Sciences}


%  --- Bibliography styling  --- %
\defaultbibliographystyle{ieeetr}
\defaultbibliography{biblio}

% --- Header --- %
\ihead{Last edited \the\year-\the\month-\the\day}
\chead{ASEN 6337 Group Project} 
\ohead{Why So Cirrus?}

% --- Starting the Document --- %
\begin{document}
\maketitle
\begin{abstract}
    \textbf{Abstract}: An efficient way to classify different types of clouds in satellite images may give insight into the Earth's atmosphere and radiation budget. Towards this end, the Max Planck Institute for Meteorology hosted an open \textit{Kaggle} competition to build the best cloud classification model. The institute provided a suite of testing images, labeled training images, and a description of the metrics that each entry would be evaluated on. This work details the development of a convolutional neural network-based classifier that achieves a Dice coefficient of XXX. 
\end{abstract}
\tableofcontents
\newpage

% -- Required Sections --- %
\section{Introduction}
\begin{wrapfigure}{r}{0.4\textwidth}
    \centering
    \includegraphics[width=0.95\linewidth]{example-image}
    \caption{A labeled training image showing all four types of clouds.}
    \label{fig:all_labels}
\end{wrapfigure}
%
This project follows the Kaggle Competition "\href{https://www.kaggle.com/competitions/understanding_cloud_organization/overview}{Understanding Clouds from Satellite Images}", hosted by the Max Planck Institute for Meteorology in 2019. The stated goal of the competition was to crowd source a way to better identify and label clouds in satellite images. The organizer provided a suite of training images, a \texttt{.csv} of the label masks for each image, and a suite of unlabeled test images. The training images are satellite observations of shallow cumulus clouds in trade wind (ocean) regions. Despite being poorly represented in climate models, these types of clouds are thought to play a significant role in the Earth's radiation balance. The organizers asked the competitors to build a model that could distinguish between four subjective patterns of clouds: \textbf{Sugar, Fish, Flower}, and \textbf{Gravel} \cite{rasp_CombiningCrowdsourcingDeep_2020, maxplanckinstituteformeteorology_UnderstandingCloudsSatellite_}; \cref{fig:all_labels} features a training image where all four classes are present. 

Provided by \href{https://worldview.earthdata.nasa.gov/}{NASA Worldview}, the \(2100 \times 1400\) pixel true-color images were acquired by two different polar-orbiting satellites.  A team of scientists selected images of three different oceanic regions, then labeled them via group consensus. Some images feature discontinuities or artifacts related to observations being stitched together from multiple passes \cite{maxplanckinstituteformeteorology_UnderstandingCloudsSatellite_}. 
%
\subsection*{Training Images}
\begin{figure}[htbp]
    \centering
    \includegraphics[width=0.23\linewidth]{example-image}
    \includegraphics[width=0.23\linewidth]{example-image}
    \includegraphics[width=0.23\linewidth]{example-image}
    \includegraphics[width=0.23\linewidth]{example-image}
    \caption{\textit{From left to right}: Images showing sugar, flower, fish, and gravel labels, respectively.}
    \label{fig:labeled_imgs}
\end{figure}
%
\begin{wraptable}{r}{0.3\textwidth}
    \centering
    \begin{tabular}{@{}c|c@{}}
    \toprule
    \begin{tabular}[c]{@{}c@{}}Number of\\ Classes Present\end{tabular} & \begin{tabular}[c]{@{}c@{}}Count of \\ Images\end{tabular} \\ \midrule
    0                                                                   & 0                                                          \\
    1                                                                   & XXX                                                        \\
    2                                                                   & XXX                                                        \\
    3                                                                   & XXX                                                        \\
    4                                                                   & XXX                                                       
    \end{tabular}
    % \caption{A table showing the number of images containing 0, 1, 2, 3, and all 4 classes. }
    \label{tab:class_ct}
\end{wraptable}
%
There are 5,546 training images provided, all of which depict clouds over water. Each image has some number of labeled regions of the four types of labels. \Cref{fig:all_labels} shows a rare training image with all types of clouds present, while \cref{fig:labeled_imgs} shows training images with only one class present in each image. 

A given image could contain any number of labeled regions belonging to each class, ranging from images with regions belonging to all four classes, to images with zero labeled regions. A summary of the number of classes per image is shown in \cref{tab:class_ct}.

\subsection*{Label Format}
The training labels are provided by the file \texttt{train.csv}, with four rows populated for each image. The first column specifies the image and the class, and the second column specifies the list of encoded pixels through run-length encoding (RLE). An example is shown in the table \cref{tab:RLE_ex}.
%
\begin{table}[h!]
    \centering
    \begin{tabular}{r|l}
    \multicolumn{1}{c|}{Image\_Label} & \multicolumn{1}{c}{EncodedPixels} \\ \hline
    \texttt{example\_img.jpg\_Fish}   & \texttt{1 3}        \\ \hline
    \texttt{example\_img.jpg\_Flower} & \texttt{10 8 22 40} \\ \hline
    \texttt{example\_img.jpg\_Gravel} &                                      \\ \hline
    \texttt{example\_img.jpg\_Sugar}  &                                     
    \end{tabular}
    \caption{An example how \texttt{example\_img.jpg} is labeled in the provided \texttt{train.csv}. The image has 3 pixels labeled as "Fish" starting at pixel 1, 8 pixels labeled as "Flower" starting at pixel 10, and 40 pixels labeled as "Flower" starting at pixel 22.}
    \label{tab:RLE_ex}
\end{table}

The RLE format references the one-indexed raveled pixel locations, starting from top to bottom and left to right. This means that the pixel \(1\) corresponds to the top-left corner, \((1,1)\), pixel \(2\) corresponds to the pixel to below pixel 1 at \((2, 1)\), etc. For each class within an image, the runs are listed in ascending order of the starting pixel. 
% 
\subsection*{Unlabeled Images and Evaluation}
There are 3,698 test images of the same resolution without labels. These test images are unlabeled and are only used to evaluate the classification model using the Dice (\(DSC\)) coefficient: 
%
\begin{equation*}
    DSC = \frac{2 | X \cap Y |}{|X| + |Y|}
\end{equation*}
%
Where \(X\) is the predicted set and \(Y\) is the ground truth. This would be unity for total agreement, and 0 for total disagreement. 

It should be noted that the submitted labels should reference the coordinates of the unlabeled images after they have been down sampled by a factor of four, giving labels to \(350 \times 525\) pixel images. Kaggle allows for late submissions of labels, giving a Dice coefficient for a submitted label list \cite{maxplanckinstituteformeteorology_UnderstandingCloudsSatellite_}. 
\section{Approach and Methodology}
\textit{This is where you should describe the data analysis approach. Formulate your remote sensing data analysis problem using methods and concepts you learned in the class. }

\subsection{Drawing from Previous Submissions}
Due to the nature of the platform, the competition's community seem to have been very collaborative and communicative. This means that many top submissions publish their codebase to reproduce their results, and discussion threads detail different approaches and code snippets that are used. We began by implementing (a few?) of the existing solutions to get an idea of the tools that people found useful, to incorporate the helper functions that have already been widely disseminated, and to better understand a diversity of approaches used.  

\subsection{Our First Approach: Divide and Concur}
As a first step towards a solution, and because of the overlapping nature of the labels, we decided to start off by creating a different classifier for each class; i.e., a sugar classifier, a gravel classifier, a flower classifier, and a fish classifier. Each member of the team was assigned a class and created a classifier on their own using \texttt{pytorch} as the neural network tool set. This allowed us to get to know these new tools and data, come up with ideas on our own, and reconvene afterwards to see if there was any convergent thinking or useful tips before synthesizing a more general model. 
%
\subsection*{James' Approach for Sugar}

\subsection{Other Ideas Considered}
Additional ideas include: 
\begin{itemize}
    \item Creating additional synthetic training images by applying different combinations of affine transformations on the supplied training images (rotating, mirroring, stretching, wrapping). 
    \item Creating a dedicated model to create labels for each class (4 models total).
\end{itemize}
\section{Results and Discussion}
\textit{This is where you should provide critical interpretation of the data analysis results. Make sure that you discuss uncertainty of your analysis results. You should back-up all of your assertions with critical arguments. This is a 
good place to put any tables or plots.}

\begin{figure}[htbp]
    \centering
    \includegraphics[width=\linewidth]{figs/training_progression.pdf}
    \caption{Showing the training loss, evaluation loss, and DICE score across training epochs.}
    \label{fig:loss}
\end{figure}
\section{Conclusions and Recommendations}
\textit{Summarize your findings and state your conclusions. What would you recommend for the future work? Any recommendations should be backed up by analysis from the previous section.}

% --- References --- % 
\bibliography{biblio.bib}
\bibliographystyle{ieeetr}

% --- Code and Such --- % 
\appendix
\section{Code Used}
\subsection{Helper Functions}
\inputminted{python}{code/kaggle_helpers.py}  
\newpage
%
\subsection{Plotting masks and Images}
A file to get to know the data, transform the data into a more usable format, and display some images and masks. This script was used to create \cref{fig:all_labels,fig:labeled_imgs}. 
\inputminted{python3}{code/investigating_data.py}  
\newpage 
\subsection{Data Reduction}
Showing K-means and PCA. This script was used to create \cref{fig:PCA}. 
\inputminted{python3}{code/pca_testing.py}  
\newpage 
\subsection{Training CNN}
\inputminted{python}{code/cloud_classr_scaled.py}
\newpage
\subsection{Evaluating CNN}
\inputminted{python}{code/NN_analysis.py}
\end{document}